\documentclass[12pt]{article}
\usepackage{amsmath,amssymb,amsfonts}
\pagestyle{plain}
\oddsidemargin 0cm
\textwidth 16cm
\topmargin -1cm
\parindent 0cm
\textheight 22cm
\parskip 3mm
\def\E{\mathbb{E}}
\def\bS{\mathbb{S}}
\def\R{\mathbb{R}}
\date{ }
\title{{University of Manchester Institute of Science and Technology}\\ \vspace*{2cm}
{\small UA351: Pure Topics II \hfill UA351}}
\author{\hspace*{-14cm}For candidates taking:\hfill \\DEGREE OF MMath\\
DEGREE OF BSc\\
FINAL EXAMINATION in HONOURS SCHOOL OF \\
PURE MATHEMATICS AND COMPUTATION\\ MATHEMATICS AND LANGUAGE STUDIES\\
MATHEMATICS\\ MATHEMATICS STATISTICS AND OPERATIONAL RESEARCH\\
MATHEAMATICS AND MANAGEMENT SCIENCES}
\begin{document}
\maketitle
\vspace*{3cm}
Date: ???????? \hfill Time: ????

\vspace*{3cm}
\begin{center}
{\large\bf Answer Three Questions}
\end{center}

\newpage
{\large\bf 1.}\\
{\large\bf (i)} Given that the set of all $3\times 3$ orthogonal
real matrices with determinant $+1$
forms a group $SO(3)$ under matrix multiplication, prove that $SO(3)$ has a
subgroup $R_Z$ consisting of rotations about the $z$-axis. Write down without proof
the definitions of two other subgroups, $R_X, \, R_Y,$ of $SO(3),$ corresponding
to rotations about the $x$-axis and $y$-axis respectively.

{\large\bf (ii)} Prove that the map
$$g:[0,2\pi]\times [-\pi/2,\pi/2]\rightarrow \E^3 :
(u,v) \mapsto (\cos v \cos u, \cos v\sin u, \sin v)$$ provides a parametrization
of the standard unit sphere $\bS^2.$

{\large\bf (iii)} Prove that $R_Z$ defines an action on the
standard unit sphere $\bS^2$ by the map $$\rho:R_Z\times \bS^2
\rightarrow \bS^2 : ([A_{ij}],[p_j]) \mapsto
[\sum_{k=1}^{k=3}A_{ik}p_k]$$ and find its orbits. Prove also that
this action is effective but neither free nor transitive.

{\large\bf (iv)} Find a parametric equation for the equator curve on $\bS^2$ in the form
$$e:[0,2\pi]\rightarrow \bS^2$$
and find the parameter value $t_0\in [0,2\pi]$ such that
$$e(t_0)=(\frac{1}{\sqrt 2},\frac{1}{\sqrt 2},0) = p, \  {\rm say}.$$
Find a parametric equation for the great circle $\gamma$ through $p$ and inclined at
angle $\pi/6$ to the equator, so that $\gamma(t_1)=p$ for some $t_1$ in the domain
of $\gamma$ and $\gamma'(t_1)$ makes an angle $\pi/6$ with $e'(t_0)$ at $p.$
What is the torsion of $\gamma?$

\newpage
{\large\bf 2.}\\
Let $\alpha :(a,b)\rightarrow \E^3$ be a regular curve with positive curvature $\kappa,$
torsion $\tau$ and a Frenet-Serret frame field $(T,N,B).$

Denote by $s:(a,b)\rightarrow \R$ the arc length function and suppose that
$\hat{\alpha} :(c,d)\rightarrow \E^3$ is a unit speed reparametrization of $\alpha.$

Denote by $\hat{\kappa}$ and
$\hat{\tau}$ the curvature and torsion respectively of $\hat{\alpha},$ and
by $(\hat{T},\hat{N},\hat{B})$ the Frenet-Serret frame field of $\hat{\alpha}.$

{\large\bf (i)} Write down without proof
the Frenet-Serret equations for $\hat{\alpha}$ and use the definitions
\begin{eqnarray*}
\kappa(t) &=& \hat{\kappa}(s(t)) \\
\tau(t) &=& \hat{\tau}(s(t)) \\
T(t) &=& \hat{T}(s(t)) \\
N(t) &=& \hat{N}(s(t)) \\
B(t) &=& \hat{B}(s(t))
\end{eqnarray*}
to obtain the Frenet-Serret equations for $\alpha.$

{\large\bf (ii)} Prove that $\alpha$ has acceleration
$$\alpha'' = s''T+(s')^2\kappa N$$
with curvature
$$\kappa = \frac{||\alpha' \times\alpha''||}{||\alpha'||^3}$$
and torsion
$$\tau = \frac{\alpha' \times\alpha''\cdot \alpha'''}{||\alpha'\times \alpha''||^2}.$$




{\large\bf (iii)} Find the curvature and torsion of the twisted cubic curve
$$\alpha: (0,1) \rightarrow \E^3 : t \mapsto (t,t^2,t^3).$$

\newpage
{\large\bf 3.} \\
Let $\Phi:M_1\rightarrow M_2$ be a local isometry between two
regular surfaces $M_1$ and $M_2$ in $\E^3$ with Gaussian curvatures
$K_1,K_2,$ respectively. A famous theorem of Gauss states
that then
$$K_1=K_2\circ\Phi.$$

Prove by counterexample that the converse is false by considering the funnel
surface $M_1$ with patch map
$$(u,v)\mapsto (v\cos u,v\sin u, \log v)$$
the helicoid $M_2$ with patch map
$$(u,v)\mapsto (v\cos u,v\sin u, u)$$
and the diffeomorphism
$$\Phi:M_1\rightarrow M_2:(v\cos u,v\sin u, \log v)\mapsto (v\cos u,v\sin u, u).$$

Do this by showing that this $M_1$ and $M_2$ have the same curvature at corresponding
points under $\Phi,$ but $\Phi$ is not an isometry.
You may use without proof the Weingarten result that the determinant of the shape
operator $S$ is given by
$$\det S = \frac{eg-f^2}{EG-F^2}$$
when the arc length formula is
$$ds^2=Edu^2+2Fdudv+Gdv^2$$
and, for a unit normal field $\hat{n}$ and a patch map $x,$ the second fundamental form
has components given by
$$e=\hat{n}\cdot x_{uu}, \ \ f=\hat{n}\cdot x_{uv}, \ \ g=\hat{n} \cdot x_{vv}.$$


\newpage
{\large\bf 4.}\\
{\large\bf (i)}
Draw a projection of an oriented trefoil knot $T,$ and number the overcrossings. From
your drawing, write down without proof
a matrix from which the Alexander polynomial ${\cal A}(T)$
may be found as the monic factor of a determinant. Find ${\cal A}(T).$

{\large\bf (ii)}
Take an identical pair of oriented trefoil knots and by joining them together
form an oriented sum knot, $R;$
take two more oriented trefoil knots, one the mirror image of
the other, and join them to form a different
summand oriented knot, $G;$ do this in such a way that $R$ and $G$ each have
a projection with six overcrossings. [In fact, one of your summands should be
the square knot and the other should be the granny knot.]
Find suitable
such projections, number the overcrossings and then compute for $R,G$ without proof
the Alexander polynomials ${\cal A}(R)$ and ${\cal A}(G).$

{\large\bf (iii)}
Write down an equation relating the three polynomials,
${\cal A}(T),$ ${\cal A}(R)$ and ${\cal A}(G).$

\end{document}
